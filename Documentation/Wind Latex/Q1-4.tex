\section{Q1-4}
What is meant by “knowledge acquisition is not sequential”? Provide a concrete example of knowledge acquisition that illustrates this.\\

By referering to the book, knowlegde acquisition is not sequential means that you cannot just pour knowledge into ones head. You might have to reconsider the knowledge learned and all experiences might not catch on as well as others.\\

For example, when developing software and delivering a prototype to the costumer, one might experience that what was first understood by the technical specification was not what the customer actually wanted. Therefore the knowledge gained was not correct and you might have to start over and therefore it is not sequential.