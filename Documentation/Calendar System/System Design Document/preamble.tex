\documentclass[12pt,a4paper,oneside,openany]{memoir} 
% Skabelon af DTU's LaTeX support gruppe, v20090423

\usepackage[utf8]{inputenc} 
%\usepackage[danish]{babel} % danske overskrifter
\usepackage[T1]{fontenc}   % fonte (output)
\usepackage{lmodern}       % vektor fonte
\usepackage{fancybox, graphicx}      % indsættelse af billeder
\usepackage{palatino}      % lækker font
\usepackage{pdfpages}      % pdf som forside evt
\usepackage{todonotes}
\usepackage{microtype}
% \linespread{1.3}           % kræver lidt mere line spacing

\newcommand{\code}[1]{\texttt{#1}}
\newcommand{\HRule}{\rule{\linewidth}{0.5mm}}

%\addto\captionsdanish{
%  \renewcommand{\contentsname}%
%    {Indholdsfortegnelse}     %
%} % Så bruger vi bare 'Indholdsfortegnelse' i stedet for 'Indhold'


\usepackage{underscore}
\usepackage{pdflscape}
\usepackage{todonotes}

\usepackage[plainpages=false,pdfpagelabels,pageanchor=false]{hyperref} % aktive links
\hypersetup{
    pdfborder = {0 0 0}
}
\def\sectionautorefname{afsnit}

\usepackage{memhfixc}  % rettelser til hyperref

\usepackage{tipa}
\pretolerance=2500     % højt tal, mindre orddeling og mere space mellem ord.
% 3000 er okey, 1000 er for lidt, 5000 i overkanten, 8000 er for meget..

\usepackage[font=small,labelfont=bf,labelsep=endash]{caption}
 
\pagestyle{headings}


\makechapterstyle{mortenovi}{%
\setlength{\beforechapskip}{0cm}%længde fra top af side til kapitel-overskrifter
\setlength{\afterchapskip}{1cm}%længde fra kapiteltekst til body-tekst
\setlength{\midchapskip}{2cm}%længe mellem kapitelnummer og kapiteltekst
\renewcommand\chapnamefont{\normalfont\Large\scshape\raggedleft}
\renewcommand\chaptitlefont{\normalfont\Huge\bfseries\sffamily}
\renewcommand\chapternamenum{}%default "kapitel"
\renewcommand\printchapternum{%
    \makebox[0pt][l]{%
    \hspace{0.4em}
    \resizebox{!}{4ex}{\chapnamefont\bfseries\sffamily\thechapter}}
    }%"kapitel. x"-linjen og dens boxe og bredder - prøv at sætte xyz ind først på de tre linjer respektivt.
\renewcommand\afterchapternum{\par\hspace{1.5cm}\hrule\vspace{0.5cm}}
\renewcommand\afterchaptertitle{\vskip\onelineskip \hrule\vskip\afterchapskip
}}
\chapterstyle{mortenovi}

\maxtocdepth{subsection} %Only parts, chapters and sections in the table of contents
\settocdepth{subsection}

% \includeonly{forord,testing} % Kompiler kun de kapitler du arbejder med.

\usepackage{listings}
\usepackage{color}

\renewcommand*\lstlistingname{Kode}

%for diagbox
\usepackage{diagbox}
\usepackage{booktabs}
% Done

\definecolor{dkgreen}{rgb}{0,0.6,0}
\definecolor{gray}{rgb}{0.5,0.5,0.5}
\definecolor{mauve}{rgb}{0.58,0,0.82}

\lstset{frame=tb, %lr
  language=Java,
  aboveskip=3mm,
  belowskip=3mm,
  showstringspaces=false,
  columns=flexible,
  basicstyle={\small\ttfamily},
  numbers=none,
  numberstyle=\tiny\color{gray},
  keywordstyle=\color{blue},
  commentstyle=\color{dkgreen},
  stringstyle=\color{mauve},
  breaklines=true,
  breakatwhitespace=true,
  basicstyle=\tiny\ttfamily
}

\usepackage{cleveref}
