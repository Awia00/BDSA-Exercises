\section{Nonfunctional requirements}

Beneath this sentence is the first draft of the non functional requirements written. It does not include, operation, interface or legal, since these subjects were irrelevant.
\\ 
\\
\HRule \\[0.4cm]
\emph{Usability} \\ 
User must know how to operate a computer and being familiar with the windows interface standard will help, but should not need any training to use the system. Documentation of how to use the program should be included an available in the program. \\
\HRule \\[0.4cm]
\emph{Reliability} \\
Should the system crash, all userdata should still be available on next launch. The system should at worst loose the data concerning an event the user was creating during the crash. Force restarting should not be acceptable and most exceptions must be caught and handled runtime.\\
\HRule \\[0.4cm]
\emph{Performance} \\
The system should be able to hold an arbitrary number of events, without slowing down. Furthermore most user actions must be near instant, and heawy work should run in the background and thereby not disturbing the user.\\
\HRule \\[0.4cm]
\emph{Supportability} \\
Updating by reinstalling is acceptable in this project, but the system should be easy to extent for a programmer.\\
\HRule \\[0.4cm]
\emph{Implementation}\\
The testing will probably be a little scarce due to the small amount of time available to develop the system, but Key components must be tested thoroughly.  The Calendar is a lightweight system, and stores data on the cloud, so there will not be any restraints on the hardware. Additionally, in it's current form the CalendarSystem is only runnable on Windows OS.\\
\HRule \\[0.4cm]

\newpage